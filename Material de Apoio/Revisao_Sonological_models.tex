Timbre é uma característica do som que não pode ser relacionada diretamente
com uma dimensão física, já que sua percepção resulta da presença (e ausência)
de diferentes propriedades sonoras. Apesar de conhecida a influência dessas
propriedades, o estudo é muito difícil, especialmente pelo fato de métodos
dependerem de respostas não ambíguas de ouvintes. Alguns estudos empregam a
noção de "classificação por similaridade" para a construção de um espaço de
timbre que, embora indiquem um caminho na busca da identificação entre causa e
sensação, não conduzem a um método claro de definição um conjunto de
coordenadas para o qual um som arbitrário pode ser diretamente mapeado.

Por outro lado, para o problema de classificação de voz, diversas técnicas de
processamento de sinal já foram aplicadas para conseguir reudção de dados
suficiente de forma a preservar a quantidade de informação necessária para
definir um mapeamento quase unívoco entre forma de onda e fonema.

Esse trabalho tenta aplicar algumas técnicas comuns de análise de sons para
sinais acústicos a fim de extrair um conjunto de parametros significantes para
a descrição e classificação de timbres de instrumentos musicais.

Análise acústica reside na modelagem matemáticos (e portanto, computacional) do
sinal (entendido como uma sequencia de amostras - samples) e no estudo de
parâmetros descritivos do modelo aplciado. O modelo deve fornecer uma
representação simplificada do conjunto de dados, ou seja, a quantidade de
parâmetros do modelo deve ser suficientemente pequena e a variância dos
parâmetros, vistos como função do tempo, devem ser significativamente
reduzidas comparadas com a variância do conjunto de dados original. Nota-se
portanto a necessidade de um processo de redução de dados. Esse processo deve
ser feito (e projetado) tendo em vista os objetivos da análise e,
naturalmente, o tipo de som que está sendo analisado. 

2.1 - 1h15min
Modelos aplicados na pesquisa de timbres podem ser divididos em três classes:

- Modelos de Sinal
- Modelos que levam em conta os mecanismos de produção de som
- Modelos que levam em conta as propriedades perceptuais auditivas

Os modelos classicos para a representação do som no estudo de timbres
normalmente empregam a representação do sinal no espaço de frequência, que tem
por base a Transformada Discreta de Fourier (DCT) em suas diversas formulações
e variantes. 

Em aplicações de reconhecimento de voz os modelos mais usados são os que
entendem a voz como um sinal e as palavras pronunciadas como um filtro sobre a
voz. Assim, tem-se um sinal com um filtro lienar. LPC e suas variantes são os
métodos mais usados nesse caso.

O último tipo de modelo, perceptual, também foi desenvolvido por grupos que
estudam processamento de voz. Uso da Transformada de Fourier de Tempo Curto
(STFT), Cepstrum, e demais métodos foram feitos levando em conta o processo
físico que caracteriza a onda obtida da transdução do sinal sonoro (variações
de pressão do ar) em um sinal elétrico.

Com o passar dos anos quase todos esses modelos foram adaptados, tendo sido
acrescidos a eles, ainda que de forma geral, fenômenos perceptuais, como por
exemplo Previsão Linear com Envelope de Frequencia (Linear Prediction on a
warped frequency scale), modelos autitivos derivados da STFT (Short-Time
Fourier Transform), anlálise preditiva de voz com base perceptual (Perceptual
based linear prediction of speech). O exemplo mais significante de modelo que
busca melhorar a análise acústica utilizando conhecimentos perceptuais é a
análise de voz por meio de Mel-frequency Cepstrum (espectro de frequencia na
escala Mel), que transforma a frequencia de um domínio linear em um domínio
logaritmico, se aproximando da forma como o sistema auditivo humano percebe a
altura (graves e agudos) dos sons. O uso de Coeficientes Mel-Ceptrais
(Mel-frequency Cepstrum Coefficients - MFCC) é quase universalmente usado para
a construção de sistemas de reconhcimento de voz automatizado. 

Todos esses modelos tem por premissa que as propriedades do sinal mudam
lentamente com o tempo, o que embora represente uma boa aproximação ainda
desconsidera mudanças sutis na dinamica do sinal sonoro. Por conta disso,
parâmetros como diferença entre parâmetros em frames sucessivos e diferença
das diferenças entre frames sucessivos passaram a estar presentes em
praticamente todos os sistemas automatizados de reconhecimento de voz.
Até a data deste trabalho (90s) essas inovações não entraram no campo da
análise de timbre, é possível que possam abrir portas para novas
interpretações e resultados na área

O uso de modelos auditivos tem ganhado vasta aceitação não apenas em pesquisas
de reconhecimento e análise de voz como também em pesquisa de timbre. Estudos
recentes apresentando modelos auditivos e modelos auditivos simplificados
trouxeram resultados promissores.

[4] [32] 

3 - Espaço de Timbre - 2h
Alguns dos parâmetros relacionado à análise acústica são parâmetros globais e
representam o som como um todo, como por exemplo a potência média. Outros
variam com o tempo e representam características de um trecho analisado em uma
curta janela de tempo. Sendo o som descrito por atributos globais e locais
(referentes a frames analisados) a grande questão é saber quais deles são mais
importantes e devem ser mantidos. Ao fim desse processo, muito provavelmente
ter-se-á um vetor \textit{m}-dimensional que representam características que variam no
tempo. O objetivo da próxima etada do processamento é, portanto, construir um
espaço com menor dimensão (2 ou 3) que represente o sinal e leve a uma
interpretação mais simples desse espaço. Esse espaço de menor dimensão será
chamado de \textit{espaço de timbre}. O espaço construído deve ser tal que
sons distintos devem ser ´´distantes" entre si e sons parecidos devem ser
``próximos" entre si, ou seja, o conceito de similaridade acústica deve estar
diretamente relacionado com o conceito de distância.

 
